\documentclass[11pt]{article}
\usepackage[letterpaper, margin=1 in]{geometry}

\usepackage{amsmath}
\usepackage{amsfonts}
\usepackage{tikz}
\usepackage{pgfplots}
\pgfplotsset{compat=1.17}
\usepgfplotslibrary{fillbetween}

\title{Generalized Linear Models}
\author{}
\date{}
\setlength{\parindent}{0pt}
\begin{document}
\maketitle
\vspace{-1.2em}
Closely related to generalized linear models (GLMs) are exponential families. An exponential family is one whose probability distribution function can be written as
$$P(y;\eta) = b(y) \exp{\left(\eta^T T(y) - a(\eta)\right)}$$
where $y$ is the data, $\eta$ is the natural parameter, $T(y)$ is a sufficient statistic, $b(y)$ is a base measure, and $a(\eta)$ is a log-partition function. As long as all functions are only dependent on the input variable, there can be any choice of definitions for $T$, $b$, and $a$ as long as the distribution function integrates to 1.

An example of a member of the exponential family is the Bernoulli distribution, which has a probability distribution function of
$$P(y;\phi) = \phi^y(1 - \phi)^{(1-y)}$$ which can then be rewritten as $$\exp{\left(\log{\left(\phi^y(1 - \phi)^{(1-y)}\right)}\right)}$$ which breaks into $$\exp{\left(y\log{\phi} + (1-y)\log{(1-\phi)}\right)}$$ and then $$\exp{\left(\log{\left(\frac{\phi}{1-\phi}\right)y+\log{(1-\phi)}}\right)}$$
which allows us to see that $b(y) = 1$, $\eta^T = \eta = \log{\frac{\phi}{1-\phi}}$, $T(y) = y$, and $a(\eta) = -\log{(1-\phi)}$. Since we want an expression in temrs of $\eta$ for $a(\eta)$, we can solve for it using $\eta = \log{\frac{\phi}{1-\phi}}$ which results in
$$\phi = \frac{1}{1+e^{-\eta}}$$
so therefore
$$a(\eta) = -\log{\left(1-\frac{1}{1+e^{-\eta}}\right)} = \log{\left(1+e^\eta\right)}$$
Some properties of exponential families are
\begin{enumerate}
    \item the MLE (maximum likelihood estimation) w.r.t $\eta$ is concave
    \item the mean $\text{E}[y;\eta]$ is equivalent to $\frac{\partial}{\partial\eta}a(\eta)$
    \item the variance $\text{Var}[y;\eta]$ is equivalent to $\frac{\partial^2}{\partial\eta^2}a(\eta)$
\end{enumerate}
\clearpage
Assume that $y \mid x ; \theta \sim \text{Exponential Family}(\eta)$ and $\eta = \theta^T x$. At test time, the output will be $\text{E}\left[y \mid x;\theta\right]$, which means that our hypothesis function is $h_\theta(x) = \text{E}\left[y \mid x;\theta\right] = \frac{\partial}{\partial \eta} a(\eta)$. \\
\vspace{0em} \\
Going back to Bernoulli's distribution, we see that $$\text{E}\left[y \mid x;\theta\right] = \frac{e^\eta}{1+e^\eta} = \frac{1}{1+e^{-\eta}}$$
which is equivalent to
$$h_\theta(x) = \frac{1}{1+e^{-\theta^T x}}$$
which is the sigmoid function used in logistic regression. We can say that logistic regression is the natural result of choosing the Bernoulli distribution for a generalized linear model.
\end{document}
